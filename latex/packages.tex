\documentclass[11pt, a4paper]{article}
% Pode diminuir a fonte para 10pt ou aumentar para 12pt

%%%%%%%%%%%%%%%%%%%%%%%---Packages---%%%%%%%%%%%%%%%%%%%%%%%
\usepackage{amsmath}
\usepackage{braket}
\usepackage{physics}
\usepackage[T1]{fontenc}
\usepackage[utf8]{inputenc}
\usepackage{lmodern}
%\usepackage{mlmodern}  % Fonte mais forte
\usepackage[top=2.5cm, bottom=2.5cm, left=3cm, right=3cm]{geometry}		% Ajusta as margens
\usepackage{graphicx}
\usepackage[english]{babel}
%\usepackage[notcite,notref]{showkeys}
\usepackage{amsmath,amssymb,mathtools,amsthm} % pacotes matematicos (amssymb inclui amsfonts)
\usepackage{xcolor} % Colorir fontes
\usepackage[colorlinks=true, bookmarksnumbered=true, bookmarksopen=true, bookmarksopenlevel=3, pdfstartview=FitH, linkcolor=black, pdfmenubar=true, pdftoolbar=true, bookmarks=true,citecolor=black, urlcolor=blue, filecolor=magenta,plainpages=false,pdfpagelabels,breaklinks]{hyperref}
\usepackage[font=footnotesize,labelfont=bf]{caption}
\usepackage{tocbasic}
\usepackage[alf]{abntex2cite}

\usepackage{braket}
\usepackage{dsfont}
\usepackage{amsmath}
\usepackage{amssymb}
\usepackage{amsthm}
\usepackage{thmtools} % for correct autoref naming
\usepackage{mathtools}
\usepackage[switch]{lineno} 
%\linenumbers
\usepackage[ruled,linesnumbered]{algorithm2e}
\SetArgSty{textnormal}
\usepackage{algpseudocode}
\usepackage{subfiles}
\usepackage{tikz}
\usepackage{minted}
\usepackage{outline}
% \usetikzlibrary{quantikz}

%%%%%%%%%%%%%%%%%%%%%%%%%%%%%%%%%%%%%%%%%%%%%%%%%%%%%%%%

\usepackage{lipsum}

\newtheorem{teorema}{Teorema}[section]				
\newtheorem*{teorema1}{Teorema}						
\newtheorem{lema}[teorema]{Lema}					
\newtheorem{corolario}[teorema]{Corol\'ario}		
\newtheorem{proposicao}{Proposi\c c\~ao}[section]	
\newtheorem{axioma}[teorema]{Axioma}				
\newtheorem{afirmacao}[teorema]{Afirma\c c\~ao}		
\theoremstyle{definition}
\newtheorem{definition}{Definition}[section]
\newtheorem{exemplo}{Exemplo}[section]			
\newtheorem{obs}{Observa\c{c}\~ao}[section]			

%%%%%%%%%%%%%%%%%%%%%%%%%%%%%%%%%%%%%%%%%%%%%%%%%%%%%%%%%%%%%%%


\DeclarePairedDelimiterX{\inp}[2]{\langle}{\rangle}{#1, #2}
\DeclareMathOperator{\ident}{\operatorname{id}}
\DeclareMathOperator{\card}{\operatorname{card}}				% Comando para cadinalidade
\DeclarePairedDelimiter\floor{\lfloor}{\rfloor}					% Comado para função piso

\renewcommand{\qedsymbol}{\textrm{Q.E.D.}}		% Adiciona Q.E.D no final da demonstração
%\renewcommand*\footnoterule{} % Tira a linha que aparece antes das notas de rodapé

\numberwithin{equation}{section}
%%%%%%%%%%%%%%%%%%%%%%%%%%%%%%%%%%%%%%%%%%%%%%%%%%%%%
